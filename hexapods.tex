\documentclass{article}

% ready for submission
\usepackage{arxiv}

% to compile a camera-ready version, add the [final] option, e.g.:
% \usepackage[final]{neurips_2018}

\usepackage[utf8]{inputenc} % allow utf-8 input
\usepackage[T1]{fontenc}    % use 8-bit T1 fonts
\usepackage{hyperref}       % hyperlinks
\usepackage{url}            % simple URL typesetting
\usepackage{booktabs}       % professional-quality tables
\usepackage{amsfonts}       % blackboard math symbols
\usepackage{nicefrac}       % compact symbols for 1/2, etc.
\usepackage{microtype}      % microtypography
\usepackage{graphicx}

\title{Why don't hexapods gallop?}

\date{July 11, 2019}

% The \author macro works with any number of authors. There are two commands
% used to separate the names and addresses of multiple authors: \And and \AND.
%
% Using \And between authors leaves it to LaTeX to determine where to break the
% lines. Using \AND forces a line break at that point. So, if LaTeX puts 3 of 4
% authors names on the first line, and the last on the second line, try using
% \AND instead of \And before the third author name.

\author{%
  Aidan Rocke\\
  \texttt{aidanrocke@gmail.com} \\
  % examples of more authors
  % \And
  % Coauthor \\
  % Affiliation \\
  % Address \\
  % \texttt{email} \\
  % \AND
  % Coauthor \\
  % Affiliation \\
  % Address \\
  % \texttt{email} \\
  % \And
  % Coauthor \\
  % Affiliation \\
  % Address \\
  % \texttt{email} \\
  % \And
  % Coauthor \\
  % Affiliation \\
  % Address \\
  % \texttt{email} \\
}

\begin{document}
% \nipsfinalcopy is no longer used

\maketitle

\begin{abstract}
   The main contribution of this paper is to explain why the overwhelming majority of hexapods don't employ the rectilinear gallop using simple mechanical arguments. 
\end{abstract}

\section{Three constraints on galloping quadrupeds:}

1. For all quadrupeds the rectilinear gallop necessarily employs one phase where its limbs are in contact with the ground and a second flight phase where its limbs are
not in contact with the ground. 

2. Given that the rectilinear gallop requires that the organism travel in approximately a straight line in a single direction there is a necessary division of mechanical labour. The forelimbs 
are specialised for traction whereas the hind limbs are specialised for propulsion. 

3. The division of mechanical labour implies that the forelimbs are in phase with each other and out of phase with the hind limbs. 

Given that four limbs are necessary for galloping and many quadrupedal mammals can gallop, four limbs are also sufficient. 

\section{Energetic costs on galloping hexapods:}

In order to demonstrate that hexapods have no advantage over galloping quadrupeds we shall compare a hexapod $H$ and a quadruped $Q$ 
of equal mass and analyse the energetic difference.  

1. In order for this hexapod to gallop as fast as a similarly-sized quadruped it must have similar limb lengths
since, according to the dynamic similarity hypothesis [2], rectilinear acceleration is approximated by: 

\begin{equation}
a \approx \frac{v^2}{L}
\end{equation}

2. Given the equal mass assumption, each limb will experience similar normal forces. Now, given that 
tensile strength is proportional to cross-sectional area, a cylindrical limb approximation yields: 

\begin{equation}
M \cdot \frac{v^2}{L} \propto R_{H} \Longrightarrow \langle R_{H} \rangle \approx \langle R_{Q} \rangle
\end{equation}

where $\langle R_{H} \rangle$ denotes the average limb radius of the hexapod.  

3. If we assume that the hexapod and quadruped have identical cylindrical limbs we may note that the rotational kinetic energy consumed by 
the hexapod, $E_{rot}^H$ is approximately: 

\begin{equation}
E_{rot}^H \approx 1.5 \cdot E_{rot}^Q
\end{equation}

since each limb must be actuated independently of the others. 

\newpage

\section{Strengths and Limitations of Hexapod locomotion}

\begin{figure}
\begin{center}
  \includegraphics[scale=0.25]{/Users/mac/Desktop/papers/Hexapod-paper/hexapod.png}
  
  
  \caption{alternating tripod gait of a hexapod}
  \label{fig:alternating tripod gait of a hexapod}
\end{center}
\end{figure}


1. \textbf{Limitations:} 

Although reductive, our comparison of the energetic profiles of hexapods vs. quadrupeds indicates that in the presence
of macroscopic forces a hexapod is unlikely to outrun a quadruped. In other words, a cheetah born with 
six limbs rather than four is unlikely to succeed in passing down its genes. That said, in 2013 three galloping beetles were 
discovered [3]. \par

The authors of [3] note that although \textit{P. endroedyi}, \textit{P. hippocrates} and \textit{P. glentoni} can walk using the normal tripod gait these 
beetles usually employ a unique galloping gait where they move each pair synchronously, stepping alternating with the front 
and middle legs. The hind legs are dragged behind even if the beetle carries no load and seem to contribute little to propulsion. \par

2. \textbf{Strengths:}

The advantage of hexapods over quadrupeds emerges when instead of emphasising speed/acceleration we focus on 
static stability. Six limbs is the smallest number of limbs that allows an organism to utilise an alternating tripod gait, which 
allows its centre of gravity to always lie within the area of support. It follows that the hexapod, unlike the quadruped, has the 
advantage of static stability at every instant of locomotion and therefore consumes less energy at low speeds. \par

\section{Discussion}
The arguments provided here are rather qualitative in nature and far from exhaustive. On the one hand, a challenge facing more detailed analyses of 
hexapod locomotion is that the overall generality of the analysis tends to disappear. On the other hand, any approach using a general 
mathematical model would necessarily be underdetermined since any given hexapod potentially has an infinite number of biomechanical
parameters. 

Considering that a definitive solution is out of reach, we may consider approaches that lead to much greater insight. One approach would
be to simulate biological evolution as a stochastic optimisation problem starting from a population of hexapods with a finite number of variable biomechanical
parameters. The aim would be to minimise an energetic cost function that corresponds to travelling in a straight line on a given surface at
a particular velocity. 

\section*{References}

\small

[1] C. Vaughan \ \& M. Malley.\ Froude and the contribution of naval architecture to our understanding of bipedal locomotion.\ 2004.

[2] Alexander RM, Jayes AS.\ A dynamic similarity hypothesis for the gaits of quadrupedal mammals.\ 1983. 

[3] Smolka et al.\ A new galloping gait in an insect.\ 2013.

\end{document}